\begin{center}
	{\Huge Introduction}
\end{center}

\section{Introduction}
Ever since computers were developed, scientists and engineers thought of artificially intelligent systems that that are mentally and/or physically equivalent to humans. In the past decades, the increase of generally available computational power provided a helping hand for developing fast learning machines, whereas the Internet supplied an enormous amount of data for training. These two developments boosted the research on smart self-learning systems, with neural networks among the most promising techniques.\newline

Human emotion analysis is a challenging machine learning task with a wide range of applications in human-computer interaction, e-learning, health care, advertising and gaming.\newline

Emotion analysis is particularly challenging as multiple input modalities, both visual and auditory, play an important role in understanding it. Given a video sequence with a human subject, some of the important cues which help to understand the user’s emotion are facial expressions, movements and activities.\newline

Most of the time there is a considerable overlap between emotion classes making it a challenging classification task.\newline

In this paper we present a deep learning based approach to modeling different input modalities and to combining them in order to infer emotion labels from a given video sequence. We mainly focus on neural network based artificially intelligent systems capable of deriving the emotion of a person through pictures of his or her face\newline

We are developing a software which can recognise the emotion of  a person from its face or audio file. We can predict the behavior of a person based on that whether he or she is happy or sad. \newline 

In total we are developing a model which can detect 7 emotions from that.

\section{System Requirements}
\textbf{Software Requirements:-}  Software requirements are the essential part to develop a project and we are using following softwares :-

\begin{enumerate}
	\item \textbf{Tensorflow} - Tensorflow is an open source software library for machine learning across a range of tasks, and developed by Google to meet their needs for systems capable of building and training neural networks to detect and decipher patterns and correlations, analogous to the learning and reasoning which humans use.
	
	Further requirements to run tensorflow with GPU support :-
	\begin{itemize}
		\item CUDA® Toolkit 8.0
		\item The NVIDIA drivers associated with CUDA Toolkit 8.0.
		\newpage
		\item cuDNN v5.1
		\item GPU card with CUDA Compute Capability 3.0 or higher
	\end{itemize}

	\item \textbf{NumPy}- NumPy is the fundamental package for scientific computing with Python. It contains among other things: a powerful N-dimensional array object. sophisticated (broadcasting) functions.
	
	\item \textbf{Python} - We need to have Python 3.0 installed in our machine in order to run our code and need an IDE for Python . Here we are using Pycharm as IDE for Python and we are developing this code in that.
	
	\item \textbf{TFLearn} - TFlearn is a modular and transparent deep learning library built on top of Tensorflow. It was designed to provide a higher-level API to TensorFlow in order to facilitate and speed-up experimentations, while remaining fully transparent and compatible with it.
	
	TFLearn features include:
	\begin{itemize}
		\item Easy-to-use and understand high-level API for implementing deep neural networks, with tutorial and examples.
		\item Fast prototyping through highly modular built-in neural network layers, regularizers, optimizers, metrics...
		\item Full transparency over Tensorflow. All functions are built over tensors and can be used independently of TFLearn.
		\item Powerful helper functions to train any TensorFlow graph, with support of multiple inputs, outputs and optimizers.
		\item Easy and beautiful graph visualization, with details about weights, gradients, activations and more...
		\item Effortless device placement for using multiple CPU/GPU.
		
	\end{itemize}

	\item \textbf{OpenCV} - OpenCV (Open Source Computer Vision) is a library of programming functions mainly aimed at real-time computer vision. The library is cross-platform and free for use under the open-source BSD license.
\end{enumerate}